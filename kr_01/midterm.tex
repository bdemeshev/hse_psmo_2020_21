\documentclass[12pt]{article}

\usepackage{tikz} % картинки в tikz
\usepackage{microtype} % свешивание пунктуации

\usepackage{array} % для столбцов фиксированной ширины

\usepackage{indentfirst} % отступ в первом параграфе

\usepackage{sectsty} % для центрирования названий частей
\allsectionsfont{\centering}

\usepackage{amsmath, amssymb} % куча стандартных математических плюшек


\usepackage{comment}

\usepackage[top=2cm, left=1.2cm, right=1.2cm, bottom=2cm]{geometry} % размер текста на странице

\usepackage{lastpage} % чтобы узнать номер последней страницы

\usepackage{enumitem} % дополнительные плюшки для списков
%  например \begin{enumerate}[resume] позволяет продолжить нумерацию в новом списке
\usepackage{caption}


\usepackage{fancyhdr} % весёлые колонтитулы
\pagestyle{fancy}
\lhead{Прикладная статистика в машинном обучении}
\chead{}
\rhead{2020-11-07, праздник статистики!}
\lfoot{}
\cfoot{НЕ ПАНИКОВАТЬ}
\rfoot{\thepage/\pageref{LastPage}}
\renewcommand{\headrulewidth}{0.4pt}
\renewcommand{\footrulewidth}{0.4pt}


\let\P\relax
\DeclareMathOperator{\P}{\mathbb{P}}

\usepackage{todonotes} % для вставки в документ заметок о том, что осталось сделать
% \todo{Здесь надо коэффициенты исправить}
% \missingfigure{Здесь будет Последний день Помпеи}
% \listoftodos --- печатает все поставленные \todo'шки


% более красивые таблицы
\usepackage{booktabs}
% заповеди из докупентации:
% 1. Не используйте вертикальные линни
% 2. Не используйте двойные линии
% 3. Единицы измерения - в шапку таблицы
% 4. Не сокращайте .1 вместо 0.1
% 5. Повторяющееся значение повторяйте, а не говорите "то же"



\usepackage{fontspec}
\usepackage{polyglossia}

\setmainlanguage{russian}
\setotherlanguages{english}

% download "Linux Libertine" fonts:
% http://www.linuxlibertine.org/index.php?id=91&L=1
\setmainfont{Linux Libertine O} % or Helvetica, Arial, Cambria
% why do we need \newfontfamily:
% http://tex.stackexchange.com/questions/91507/
\newfontfamily{\cyrillicfonttt}{Linux Libertine O}

\AddEnumerateCounter{\asbuk}{\russian@alph}{щ} % для списков с русскими буквами
\setlist[enumerate, 2]{label=\asbuk*),ref=\asbuk*}

%% эконометрические сокращения
\DeclareMathOperator{\Cov}{Cov}
\DeclareMathOperator{\plim}{plim}

\DeclareMathOperator{\Corr}{Corr}
\DeclareMathOperator{\Var}{Var}
\DeclareMathOperator{\E}{E}
\def \hb{\hat{\beta}}
\def \hs{\hat{\sigma}}
\def \htheta{\hat{\theta}}
\def \s{\sigma}
\def \hy{\hat{y}}
\def \hY{\hat{Y}}
\def \v1{\vec{1}}
\def \e{\varepsilon}
\def \he{\hat{\e}}
\def \z{z}
\def \hVar{\widehat{\Var}}
\def \hCorr{\widehat{\Corr}}
\def \hCov{\widehat{\Cov}}
\def \cN{\mathcal{N}}


\begin{document}

Дорогой храбрый воин или храбрая воительница! Удачи тебе на малом празднике по прикладной статистике!
Начни с того, что напиши клятву и подпишись под ней:

\vspace{10pt}
\textit{Я клянусь честью студента, что буду выполнять эту работу самостоятельно.}
\vspace{10pt}


А теперь — задачки:


\begin{enumerate}
\item 
   
   Компания «ГолденАльп» тестирует два новых вкуса шоколада: с орешками и солёной карамелью. 
   Фокус-группа разбивают на две непересекающиеся части: $N_1$ человек пробуют шоколад с орешками, 
   а $N_2$ — с солёной карамелью. 
   
   Каждый участник пробует лишь один тип шоколада и одобряет или не одобряет опробованный вкус.

   Пусть $X_1$ — число человек, одобривших шоколад с орешками, 
   а $X_2$ — одобривших шоколад с солёной карамелью. 
   
   Будем предполагать, что $X_1 \sim \mathrm{Bin}(N_1, p_1)$, $X_2 \sim \mathrm{Bin}(N_2, p_2)$.
   
   Руководство компании «Голден Альп» хочет узнать, есть ли основание полагать, 
   что один вкус шоколада предпочитается другому.

   Для этого её статистический отдел предлагает исследовать величину $p = p_1 - p_2$.
   
   \begin{enumerate}
   	\item Найди $\hat{p}_{ML}$.
   	\item Построй $95\%$ доверительный интервал для $p$.
   	\item Подробно опиши, как построить $95\%$ доверительный интервал для $p$ при помощи какого-нибудь из методов бутстрэпа (метод выбирай сам).
	   \item По результатам эксперимента оказалось, что $N_1 = N_2 = 500$, $X_1 = 400$, $X_2 = 390$. 
	   Сформулируй гипотезу, которая позволит ответить на вопрос руководства компании. 
	   Протестируй эту гипотезу при помощи $LR$ и $LM$ тестов на уровне значимости $5\%$.
   	% \item Есть ли основание полагать, что один вкус шоколада предпочитается другому?
   \end{enumerate}
	%\begin{flushright}
	%	\textit{\small (По мотивам: Wasserman, All Of Statistics)}
	%\end{flushright}

	
	\item 
	
	Пусть $X$ и $Y$ — $n$-мерные случайные векторы, 
	образованные из независимых одинаково распределённых случайных величин. 
	
	Докажи, что $D_{KL}(p_X, p_Y) = nD_{KL}(p_{X_1}, p_{Y_1})$, где $p$ — функция плотности.
	
	\item 
	
	Докажи асимптотическую эквивалентность тестов $LR$ и $W$ в следующем смысле:
	\[
	\plim_{n \to \infty} \frac{W}{LR} = 1.
	\]
	
	\textit{Не знаешь с чего начать? Разложи логарифм правдоподобия по Тейлору и найди приблизительное выражение для $LR$-статистики}.
	

	\vspace{20pt}

	О храбрый воин и храбрая воительница! На следующей страничке есть ещё задачки! 

	\newpage
	\item Исходная выборка $y$ — вектор из $n$ независимых случайных величин, 
	равновероятно принимающих значения $0$ и $1$. Пусть $y^*$ — одна из бутстэп-выборок.

	\begin{enumerate}
		\item Просто для удобства выпиши $\E(y_i)$, $\Var(y_i)$, $\E(\bar y)$, $\Var(\bar y)$.
		\item Найди $\E(y^*_i)$, $\Var(y^*_i)$, $\E(\bar y^*)$, $\Var(\bar y^*)$.
		\item Найди $\Cov(y_i, y_i^*)$, $\Cov(\bar y, \bar y^*)$.
	\end{enumerate}

	
	\item 

	У меня есть три монетки. 
	Они выпадают орлом с вероятностями $p_1$, $p_2$ и $p_3 = 1$.  
	Я провожу эксперимент из $100$ раундов.

	В каждом раунде я равновероятно выбираю одну из монеток. 
	Подбрасываю её два раза и записываю число выпавших орлов. 

	После окончания эксперимента у меня остаётся на бумажке $100$ записанных чисел. 
	Какая монетка подкидывалась в каждом раунде, я не помню. 

	Опиши EM-алгоритм для оценивания неизвестных $p_1$ и $p_2$.

	Если формулы для какого-то шага выводятся в явном виде, то выведи их.
	Если формулы для какого-то шага не выводятся в явном виде, то объясни, 
	какая оптимизационная задача будет решаться численно. 


\item 

Рассмотрим модель множественной регрессии $y = X\beta + u$, оцениваемую при помощи МНК. 
Число наблюдений равно $n=500$, число регрессоров равно $k=10$, включая константный.
Все регрессоры ортогональны друг другу.
\begin{enumerate}
	%\item Докажите, что система $X^TX\hat{\beta} = X^Ty$ всегда имеет решение. 
	%Длину какого вектора минимизирует решение этой системы?
	\item Мелиодас строит оценку МНК по константному и первым следующим за ним четырем регрессорам, 
	а Элизабет строит оценку МНК по константному и оставшимся пяти регрессорам. 
	Покажи на \textit{единой} картинке МНК $\hat{y}$, $TSS$, $ESS$, $RSS$ и $R^2$ в их регрессиях.

	\item Эсканор строит оценку МНК по всем $10$ регрессорам. 

	Покажи на картинке МНК из предыдущего пункта $\hat{y}$, $TSS$, $ESS$, $RSS$ и $R^2$ в его регрессии.

	\item Как соотносятся $R^2$ в регрессиях Эскарнора, Мелиодаса и Элизабет? 
	% Прав ли Эсканор? Объясни интуитивно.
	%\item Пусть теперь нам известно, что $u \sim \mathcal{N}(0, I)$ и что все $\beta_i = 0$, 
	%то есть истинная зависимость на самом деле выглядит как $y = u$. 
	%Однако Диана не знает об этом и оценивает регрессию $\hat{y} = X\hat{\beta}$ при помощи МНК. 
	%Какое распределение будут иметь величины $\Vert y \Vert^2$, $\Vert \hat{y} \Vert^2$, $TSS$, $ESS$, $RSS$? 
\end{enumerate}


\end{enumerate}

\end{document}