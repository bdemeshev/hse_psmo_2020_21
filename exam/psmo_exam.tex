\documentclass[12pt]{article}

\usepackage{tikz} % картинки в tikz
\usepackage{microtype} % свешивание пунктуации

\usepackage{array} % для столбцов фиксированной ширины

\usepackage{indentfirst} % отступ в первом параграфе

\usepackage{sectsty} % для центрирования названий частей
\allsectionsfont{\centering}

\usepackage{amsmath, amssymb} % куча стандартных математических плюшек


\usepackage{comment}

\usepackage[top=2cm, left=1.2cm, right=1.2cm, bottom=2cm]{geometry} % размер текста на странице

\usepackage{lastpage} % чтобы узнать номер последней страницы

\usepackage{enumitem} % дополнительные плюшки для списков
%  например \begin{enumerate}[resume] позволяет продолжить нумерацию в новом списке
\usepackage{caption}


\usepackage{fancyhdr} % весёлые колонтитулы
\pagestyle{fancy}
\lhead{Прикладная статистика в машинном обучении}
\chead{}
\rhead{2020-12-21, праздник статистики!}
\lfoot{}
\cfoot{НЕ ПАНИКОВАТЬ}
\rfoot{\thepage/\pageref{LastPage}}
\renewcommand{\headrulewidth}{0.4pt}
\renewcommand{\footrulewidth}{0.4pt}


\let\P\relax
\DeclareMathOperator{\P}{\mathbb{P}}

\usepackage{todonotes} % для вставки в документ заметок о том, что осталось сделать
% \todo{Здесь надо коэффициенты исправить}
% \missingfigure{Здесь будет Последний день Помпеи}
% \listoftodos --- печатает все поставленные \todo'шки


% более красивые таблицы
\usepackage{booktabs}
% заповеди из докупентации:
% 1. Не используйте вертикальные линни
% 2. Не используйте двойные линии
% 3. Единицы измерения - в шапку таблицы
% 4. Не сокращайте .1 вместо 0.1
% 5. Повторяющееся значение повторяйте, а не говорите "то же"



\usepackage{fontspec}
\usepackage{polyglossia}

\setmainlanguage{russian}
\setotherlanguages{english}

% download "Linux Libertine" fonts:
% http://www.linuxlibertine.org/index.php?id=91&L=1
\setmainfont{Linux Libertine O} % or Helvetica, Arial, Cambria
% why do we need \newfontfamily:
% http://tex.stackexchange.com/questions/91507/
\newfontfamily{\cyrillicfonttt}{Linux Libertine O}

\AddEnumerateCounter{\asbuk}{\russian@alph}{щ} % для списков с русскими буквами
\setlist[enumerate, 2]{label=\asbuk*),ref=\asbuk*}

%% эконометрические сокращения
\DeclareMathOperator{\Cov}{Cov}
\DeclareMathOperator{\cov}{\Cov}
\DeclareMathOperator{\plim}{plim}

\DeclareMathOperator{\Corr}{Corr}
\DeclareMathOperator{\Var}{Var}
\DeclareMathOperator{\E}{E}
\def \hb{\hat{\beta}}
\def \hs{\hat{\sigma}}
\def \htheta{\hat{\theta}}
\def \s{\sigma}
\def \hy{\hat{y}}
\def \hY{\hat{Y}}
\def \v1{\vec{1}}
\def \e{\varepsilon}
\def \he{\hat{\e}}
\def \z{z}
\def \hVar{\widehat{\Var}}
\def \hCorr{\widehat{\Corr}}
\def \hCov{\widehat{\Cov}}
\def \cN{\mathcal{N}}


\begin{document}

Дорогой храбрый воин или храбрая воительница! Удачи тебе на большом празднике по прикладной статистике!
Начни с того, что напиши клятву и подпишись под ней:

\vspace{10pt}
\textit{Я клянусь честью студента, что буду выполнять эту работу самостоятельно.}
\vspace{10pt}


А теперь — задачки:



	\begin{enumerate}
		\item  Известно, что в среднем за час в Ромашково прибывает $\lambda$ паровозиков. Дежурный по станции, во всём придерживающийся байесовского подхода, решил оценить параметр $\lambda$. Для этого он собрал очень большую выборку $X_1$, $\ldots$, $X_n$ моментов прибытия паровозиков. Не будем держать интригу и сразу скажем, что $X_i \sim \mathrm{Pois}(\lambda)$ и все $X_i$ независимы.
	\begin{enumerate}[label=\alph*)]
		\item Пусть $\lambda \sim \Gamma(\alpha, \beta)$. Покажите, что апостериорное распределение $\lambda$ также является гамма-распределением.
		
		\textit{Напоминание:} плотность гамма-распределения имеет вид
		\[
		f(x) = \dfrac{\alpha^{\beta}x^{\beta - 1}}{\Gamma(\beta)}e^{-\alpha x}
		\]
		при $x \in (0, +\infty)$, $\alpha > 0$, $\beta > 0$.
		
		\item Постройте 95\%-ый байесовский доверительный интервал для $\lambda$.
		
		\item Выведите априорное распределение Джеффриса. Используя его, выведите апостериорное распределение $\lambda$.
		
		\item Для любого из предыдущих пунктов выведите в явном виде какие-нибудь две возможные точечные байесовские оценки для $\lambda$.
		
		% \item Опишите, как можно получить апостериорное распределение $\lambda$ при помощи алгоритма Метрополиса-Гастингса.  
	\end{enumerate}


\item Исследовательница Кларисса считает, что в модели 
\[
y_i = \beta_0 + \beta_1 x_i + \varepsilon_i
\]
имеется гетероскедастичность следующего вида: $\Var(\varepsilon_i) = \exp(\alpha_0 + \alpha_1 x_i)$.

\begin{enumerate}[label = \alph*)]
	\item Скорректируйте гетероскедастичность и выведите формулу эффективной оценки в явном виде. 
	\item Поясните, как построить доверительный интервал, устойчивый к гетероскедастичности, используя стандартные ошибки Уайта.
	\item Сформулируйте гипотезу о гомоскедастичности и найдите оценки неизвестных параметров в предположении о гомоскедастичности методом максимального правдоподобия.
\end{enumerate}


\newpage
	
	\item Неаккуратный исследователь Иннокентий хочет оценить следующую линейную модель:
	\[
	y_i = \beta_0 + \beta_{x}X_{i} + \beta_zZ_{i} + \beta_mM_{i} + u_i
	\]
	при помощи МНК. Иннокентий считает, что все регрессоры являются стохастическими с математическим ожиданием $\mu_j$ и дисперсией $\sigma^2_j$, $j \in \{x, z, m\}$. Внутренний голос говорит Иннокентию, что все регрессоры независимы между собой и со случайной ошибкой, кроме $M_i$: $\cov(M_i, u_i) \ne 0$.
	
	К сожалению, при сборе данных Иннокентий часто отвлекался, а потому получилось, что
	\begin{itemize}
		\item Был получен не $X_i$, а $X_i^* = X_i + \alpha$, $\alpha$ — константа.
		\item Был получен не $Z_i$, а $Z_i^* = Z_i + \nu$, $\nu$ — случайная величина с математическим ожиданием 0 и дисперсией $\sigma^2_{\nu}$.
	\end{itemize}

	Иннокентий не заметил ошибок при сборе данных, а потому оценивает регрессию
	\[
	\hat{y}_i = \hat{\beta}_0 + \hat{\beta}_{x}X^*_{i} + \hat{\beta}_zZ^*_{i} + \hat{\beta}_mM_{i}.
	\]
	
	\begin{enumerate}[label = \alph*)]
		\item Найдите предел при вероятности оценок $\hat{\beta}_{x}$, $\hat{\beta}_{z}$, $\hat{\beta}_{m}$. Прокомментируйте, является ли каждая из оценок состоятельной.
		\item Для каждой несостоятельной оценки из предыдущего пункта предложите корректировку, которая сделала бы её состоятельной.
		\item Иннокентий подозревает, что в модели есть проблема эндогенности. Проведя в поисках четыре дня, Иннокентий нашёл четыре переменные $Q_i$, коррелирующие со всеми регрессорами в его модели и при этом не зависимые от случайной ошибки. Выведите оценки двухшагового МНК для модели Иннокентия.
		% \item Изобразите на единой картинке оценки обычного МНК и оценки обоих шагов двухшагового МНК.
		% \item Победив эндогенность, Иннокентий начал переживать о том, нет ли в его модели мультиколлинеарности. Подробно опишите, как Иннокентию проверить наличие этой проблемы и предложите метод её решения.
	\end{enumerate}

	
	
	\item Исследователь Винни-Пух использует две модели, описывающие вектор $y=(y_1, y_2, \ldots, y_n)$. 
	Одна модель подсказана Совой, вторая — Кроликом. 
	Как известно, у Винни-Пуха опилки в голове, поэтому обе модели содержат $k=0$ параметров.

	Величины $y_i$ в обеих моделях и в реальности независимы и одинаково распределены.

	Докажите, что величина $\hat \Delta = (AIC_{\text{Кролик}} - AIC_{\text{Сова}})/2$ состоятельно оценивает $\Delta = KL(p||p_{\text{Кролик}}) - KL(p||p_{\text{Сова}})$.
	
	Здесь $p$ — реальное распределение вектора $y$, а $p_{\text{Кролик}}$ и $p_{\text{Сова}}$ — модельные.

\newpage
\item Исследовательница Мадлен проводит снижение дисперсии с помощью преобразования CUPED:
\[
X_{\text{cuped}} = 	X_{\text{post}} - \theta (X_{\text{pre}} - \E(X_{\text{pre}})).
\]

Здесь $X_{\text{post}}$ — метрика после начала эксперимента, а $X_{\text{pre}}$ — метрика до начала эксперимента.

\begin{enumerate}
	\item Явно напишите, какая целевая функция оптимизируется при подборе $\theta$.
	\item Выведите формулу для оптимального $\theta$. 
	\item Постройте график зависимости отношения дисперсий $\Var(X_{\text{cuped}})/\Var(X_{\text{post}})$ от корреляции $\rho = \Corr(X_{\text{pre}}, X_{\text{post}})$.
\end{enumerate}


\item Априорно исследователь Аверкий считает, что вероятность дождя имеет ожидание равное $1/3$ и дисперсию $1/32$. 

Затем Аверкий выбирает 10 случайных дней и 5 из них оказываются дождливыми.

\begin{enumerate}
	\item Выберите подходящее удобное априорное распределение.
	\item Постройте 90\%-ый апостериорный байесовский интервал для вероятности дождя.
	\item Друг Аверкия Аркадий считает, что вместо байесовского подхода можно было получить тот же результат в рамках классического подхода. 
	Аркадий предлагает заменить априорное распределение на дополнительные фиктивные наблюдения и использовать метод максимального правдоподобия.
	Сколько фиктивных дней наблюдений нужно добавить, и сколько из них должны быть дождливыми, чтобы точечная оценка Аркадия 
	совпала с апостериорным ожиданием Аверкия?
\end{enumerate}



\end{enumerate}



\end{document}